Die grundsätzliche Motivation der Arbeit ist die Entwicklung einer Navigationstechnik von Gruppen über große Distanzen, also im Nicht-Vista-Space, wobei sich alle Gruppenmitglieder in einem kohärenten Workspace befinden. Zu Beginn werden Kriterien analysiert, die eine gute Navigationstechnik ausmachen und dabei Metriken festgelegt, nach denen die zu entwickelnde Technik evaluiert werden soll.
Durch eine Untersuchung verwandter Arbeiten, die sich mit Navigation über große Distanzen und mit mehreren Nutzern sowie dem Raumverständnis und der Gruppeninteraktion befassen, werden Konzepte abgeleitet, welche zur Entstehung dieser Technik beitragen.
Nachdem sich die Navigation mit Hilfe sogenannter \textit{World in Miniatures} in vielerlei Hinsicht als geeignet erweist, werden in einem anschließenden Interaktionsdesign Möglichkeiten vorgestellt, wie man eine davon abgeleitete Technik für Mehrbenutzer umsetzen kann.
Im Anschluss wird kurz die Umsetzung der geeignetsten Ausführung beschrieben. Durch eine anschließende Evaluierung in Form einer Studie wird gezeigt, dass die Technik die in der Anforderungsanalyse bestimmtem Kriterien erfüllt und die zuvor aufgestellten Hypothesen erfüllt werden. Zum Abschluss werden Möglichkeiten diskutiert, wie die entwickelte Technik weiter optimiert werden kann.