Die grundsätzliche Motivation dieser Arbeit ist die Entwicklung einer Interaktionstechnik zur Gruppennavigation über große Distanzen, also jenseits direkt sichtbarer Ziele (Nicht-Vista-Raum, vgl. \cite{montello1993scale}). Nach einer Einführung in die Grundlagen und spezifischen Anforderungen an Mehrbenutzer Virtual Reality, werden Kriterien analysiert, die eine gute Navigationstechnik ausmachen.
Auf Grundlage von Erkenntnissen aus verwandten Arbeiten, werden Interaktionskonzepte abgeleitet, die eine bequeme und effektive Gruppennavigation über große Distanzen erlauben und gleichzeitig zum räumlichen Verständnis der explorierten virtuellen Umgebung beitragen.
Die Nutzung zusätzlicher Überblicksdarstellungen, sogenannter \textit{World in Miniatures} (WIM) eignet sich dabei in vielerlei Hinsicht.
Basierend auf einer weiteren Analyse von Forschungsergebnissen zur Navigation und räumlicher Orientierung in realen und virtuellen Umgebungen werden Hypothesen für eine heuristische Evaluierung der vorgeschlagenen Navigationstechnik aufgestellt. Im Rahmen der Arbeit wurde auch eine Anwendung zur Durchführung heuristischer und vergleichender Studien implementiert. Ihre Funktionalität wurde in einer Pilotstudie getestet. Zum Abschluss werden Möglichkeiten diskutiert, wie die entwickelte Technik weiter optimiert werden kann.