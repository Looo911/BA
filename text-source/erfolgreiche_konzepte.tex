Welche der bis jetzt vorgestellten Konzepte können also zur Umsetzung der geplanten Technik dienlich sein? An dieser Stelle sollen die bisherigen Erkenntnisse zu erfolgreichen Konzepten zusammengefasst werden, sodass im Anschlusskapitel die genaue Umsetzung der Technik designt werden kann.
Zuallererst steht das Konzept der \textbf{WIM}: Wie in den letzten beiden Kapiteln beschrieben, kann diese zur Navigation, Interaktion und Raumwahrnehmung dienen. 
Unsere Technik soll also diese verwenden, was mit sich bringt, dass unsere Technik weder \textit{Physical Walking}, noch \textit{Manual Viewpoint Manipulation} und kein \textit{Steering} verwendet, höchstens zum Nachjustieren im Vista Raum. \textit{Route Planning} wäre in einer WIM denkbar, um mit Touristen eine gemeinsame Route zu beschreiten, allerdings ist dies eher für kurze Distanzen praktikabel und widerspricht unserer Absicht, \textbf{Motionflow zu vermeiden}. Daher wird der \textit{Target-based Travel} in Form eines \textbf{Teleports} das Mittel der Wahl sein. Dabei wird die Nutzergruppe, welche sich in einem \textbf{kohärenten Workspace} befindet zuerst in der WIM platziert, woraufhin der Teleport zu diesem Ort stattfindet.
Die Nutzergruppe soll dabei visualisiert werden, sodass der Navigator den Ort eines jeden Teilnehmers nach der Reise kennt. Damit können einerseits \textbf{Kollisionen ausgeschlossen} werden, andererseits kann sichergestellt werden, dass ein etwaiger Point of Interest \textbf{im Sichtbereich aller Reisenden} liegt. Auch der \textbf{aktuelle Standort} der Gruppe vor einer Reise muss in der WIM sichtbar sein, sodass die Nutzergruppe Raumbeziehungen wahrnehmen und lernen kann. Die WIM muss \textbf{für alle sichtbar} sein, wobei sie gedreht werden kann, um verdeckte oder entfernte Ziele zu erreichen. Die Orientierung soll allerdings regelmäßig zurückgesetzt werden, sodass Nutzer in der Lage sind, sich ein kognitives Abbild der Karte zu erstellen.  Weiterhin müssen alle wichtigen Aktionen (wie z.B. das Platzieren der Nutzergruppe) für alle Nutzer zu jeder Zeit sichtbar sein, dafür dient das Konzept des \textbf{See Through} \cite{Argelaguet2011See-throughReality}. Hiermit können wichtige Referenzpunkte, die Nutzern ansonsten durch virtuelle Objekte verdeckt wären, sichtbar gemacht werden.
Der Ablauf einer Reiseplanung muss neben der Beobachtbarkeit auch jederzeit \textbf{unterbrechbar} und \textbf{umkehrbar} sein. Für den Teleport gilt: Jede Transition muss rechtzeitig erkennbar sein und kein Nutzer darf von dieser überrascht werden.
Um dem entgegenzuwirken, empfiehlt sich eine Art \textbf{Voransicht} (oder \textit{Peephole}) ähnlich der Photoportale, mit dem die Nutzer bereits einschätzen können, wo sie landen werden. Auch der Übergang, bei dem sich die Voransicht solange vergrößert, bis die Nutzer tatsächlich zu diesem Ort teleportiert werden, erscheint eine vielversprechende Vorgehensweise zur Vermeidung von Unwohlsein, die in unserer Technik Anwendung finden soll.
Da Nutzer von einer Vielzahl an Funktionen, die sich auf viele Knöpfe verteilen, häufig überfordert sind, bis sie eine gewisse Übung mit diesen haben, soll unsere Technik mit möglichst einem Knopf bedienen lassen und die Bedeutung der jeweiligen Aktion durch die dazugehörige Geste beschrieben werden, sodass man im besten Fall schon durch zusehen in der Lage ist, die Technik zu bedienen.

Die von Pierce et al. (\cite{pierce_representations}) vorgeschlagene Vorgehensweise, sich durch einen hierarchisch organisierten Baum zu navigieren, um schnell zwischen Orten zu wechseln erscheint als effektiv und sehr praktikabel. Deswegen soll unsere Technik als Erweiterung zu dieser Vorgehensweise dienen: Bis zum Punkt, an dem man am untersten Level des Baums, also in der Nachbarschaft angekommen ist, soll unsere Technik analog zu dieser funktionieren. Im untersten Baumlevel angekommen, knüpft unsere Technik also an und dient der schnellen Platzierung in der Nachbarschaft. Deshalb wird von einer Implementierung der Baumtraversion im Umfang dieser Arbeit abgesehen.