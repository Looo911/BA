Um eine geeignete Technik zu finden, mit der sich mehrere Nutzer gemeinsam fortbewegen können, bedarf es vorher festgelegter Metriken mit denen sich eine Technik als gut oder nicht gut, geeignet oder ungeeignet klassifizieren lässt.
Bowman et al. \cite{Bowman1998AEnvironments} liefern hierfür eine Liste an Performanz Metriken, die eben dieses leistet:

\begin{addmargin}[25pt]{0pt} 
\textbf{Speed / Task Completion Time}: (Geschwindigkeit / Bearbeitungszeit)\\
\textbf{Accuracy} (Genauigkeit) \\
\textbf{Spatial Awareness} (Raumwahrnehmung) \\
\textbf{Ease of Learning} (Lernbarkeit) \\
\textbf{Ease of Use} (Nutzbarkeit) \\
\textbf{Information Gathering} (Sammeln von Informationen) \\
\textbf{Presence} (Präsenz) \\
\textbf{User Comfort} (Nutzerkomfort) \\
\end{addmargin}
 
Ein beispielhafter Use Case kann helfen, diese Metriken in unserem Kontext zu verstehen:
Man stelle sich vor, eine Nutzergruppe (wie beispielsweise die Tourismusgruppe aus dem einleitenden Beispiel) stände in einer virtuellen Umgebung vor einem Monument, zum Beispiel dem Haupthaus der Bauhaus Universität in Weimar, und will infolgedessen zu einem anderen Platz gelangen, welcher sich nicht in der Nähe befindet. Ein (eher nahegelegenes) Ziel wäre z.B. das Denkmal von Goethe und Schiller vor dem deutschen Nationaltheater in Weimar oder der (sehr viel entferntere) Fernsehturm in Berlin
Die Zeit vom Beginn der Reise, über das Auswählen des Ziels, sowie der Reise und dem Ankommen am Ziel sind dabei im erste Kriterium \textbf{(1: Task Completion Time)} repräsentiert.\\
Die Präzision \textbf{(2: Accuracy)} beschreibt dabei, wie exakt die Reisegruppe am gewünschten Ziel ankommt. Das heißt, ob die “Touristen” nach der Reise genau vor dem Fernsehturm steht oder nur nahegelegen im Berliner Zentrum. Ist die Technik dagegen sehr präzise, kann die Reisegruppe entscheiden, wo genau sie vor oder in dem Fernsehturm steht. Dies würde aber zumeist bedeuten, dass sie aufwendiger zu benutzen ist \textbf{(5: Ease of Use)}, da eine so exakt geplante Reise viel aufwändiger durchzuführen ist, als eine grobe Zielauswahl.\\
Für die Nutzbarkeit spielt aber auch das Interaktionsdesign der entsprechenden Technik eine wichtige Rolle, welches, wenn einfach gestaltet, die Technik leichter zu benutzen macht. Im besten Fall gilt dies auch für Einsteiger, sodass auch ungeübte Nutzer schnell eine entsprechende Navigationsabfolge planen und durchführen können \textbf{(4: Ease of Learning)}.\\
Das Kriterium des Sammelns von Informationen während der Reise (\textbf{6: Information Gathering)} würde in unserem Kontext beschreiben, wie viel der Nutzer während der Reise gelernt hat. Solche Informationen wären z.B. der genaue Weg vom Bauhaus zum Theater oder die genutzten Autobahnen und deren angrenzenden Großstädte auf dem Weg nach Berlin.\\
Das Kriterium der räumlichen Wahrnehmung \textbf{(3: Spatial Awareness)} wiederum bestimmt, wie gut Nutzer ihre aktuelle Position im Raum nach der Reise einschätzen können. Wissen die Nutzer, in welche Richtung sie nach des Reise sehen, wo sie sich auf dem Globus befinden und aus welcher Himmelsrichtung sie gekommen sind?\\
Die Präsenz \textbf{(7: Presence)} beschreibt dabei das Gefühl der Immersivität der Reise, also das Gefühl, dass die virtuelle Umwelt um Ihnen herum “real” wirkt. Wird während der Reise, durch Aufgaben oder Abläufe die Konsistenz der virtuellen Umgebung gebrochen, wird diese Metrik verletzt.\\
Der letzte Punkt beschreibt, wie gut sich die Technik auf das Wohlbefinden des Nutzers \textbf{(8: User Comfort)} auswirkt. Hierbei sollte sichergestellt werden, dass es keinem Nutzer durch das Nutzen der Reisetechnik schwindelig oder übel wird, denn in vielen Fällen bringt die Fortbewegung in virtuellen Umgebungen ein Unwohlsein mit sich.

Da ein Schwerpunkt dieser Arbeit jedoch auf dem Reisen als Gruppe liegt, bedarf es eines weiteren Kriteriums, welches Bowman et al. im Hinblick auf Einzelnutzernavigation nicht berücksichtigten, nämlich den Aspekt der Gruppeninteraktion \textbf{(9: Group Interaction)}. Wie gut die Gruppen interagieren kann, ob jeder involviert wird und für alle die schon genannten Kriterien gleichermaßen gelten, sind Aspekte, die in diesem Punkt zu berücksichtigen sind.
So wäre das gemeinsame Einverständnis zum “Antritt” sowie das gemeinsame Planen der Reise, Punkte deren Nutzbarkeit im Gruppengefüge untersucht werden müssen.

Auf all diese Kriterien soll auch bei der Entwicklung unserer Technik eingegangen werden. Wie genau sie in verwandten Arbeiten gemessen werden und in welcher Form diese Quantisierung in unserem Fall stattfindet, wird in der Beschreibung der Studie im 9. Kapitel näher beschrieben.