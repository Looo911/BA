Mit der zunehmenden Popularisierung immersiver, virtueller Umgebungen werden die Ansprüche an verschiedene Arten, sich durch jene zu bewegen, zunehmend vielfältiger.
Für einzelne Benutzer hat sich im Konsumentenbereich zumeist die Teleportation in den meisten Fällen gegenüber sogenannter “Steering Navigation”\footnote{\textit{Steering Navigation}: Die Navigation durch den 3D-Raum mit 6 Freiheitsgraden, genauere Erklärung in Kapitel 4} durchgesetzt und ist in vielen Anwendungen, wie Computerspielen, zum Standard geworden. 
Dennoch gibt es viele Szenarien, in denen weiter gedacht werden muss, wenn es sich zum Beispiel um die Navigation über große Distanzen dreht oder die Größenverhältnisse in der Umgebung während dem Navigieren verändert werden, wobei man doch immer wieder an der Problematik der Größe des Workspaces stößt, welche an die Maße des realen Raumes, in dem man sich befindet, gebunden ist.
Dies kann zu vollständig unterschiedlichen Ansprüchen an die Navigation führen: der Arzt, der  beispielsweise mit einer virtuellen Repräsentation eines Organs an seinem Arbeitsplatz arbeitet, braucht andere Techniken, als ein Nutzer, der sich zum Zwecke des eigenen Entertainments, durch die Welt von Google Earth navigiert und sich schnell zwischen den Metropolen (oder Heimatdörfern) der Welt bewegen will. Eine bis jetzt wenig betrachtete Besonderheit bei diesen verschiedenen Fortbewegungstechniken ist dabei die Navigation einer ganzen Gruppe von Nutzern, welche aber gerade in Zukunft eine wichtige Rolle in virtuellen Umgebungen spielen dürfte. Der Fremdenführer, der vor der “realen” Führung einen virtuellen Überblick über eine Stadt geben will oder ein Architekt, der mit Mitarbeitern verschiedene Standorte für ein Gebäude virtuell erkunden will, sind nur zwei Beispiele der Fülle an möglichen Anwendungsszenarien.
Ziel dieser Arbeit ist die Entwicklung und Evaluierung einer Technik, mit der es mehreren Nutzern möglich ist, gemeinsam durch virtuelle Umgebungen verschiedener Art zu reisen und dabei den wichtigsten Kriterien, die die Qualität einer Navigationstechnik bestimmen, entspricht. Nach einer Einführung in die kollaborative VR, soll analysiert werden, welche diese Kriterien sind und wie diese gemessen werden können.
Daraufhin sollen aktuelle Arbeiten zum Thema “Navigation in virtuellen Umgebungen” vorgestellt werden und auf ihre Anwendbarkeit in Mehrbenutzerszenarien untersucht werden, wobei dabei der Fokus auf das schnelle und effektive Reisen, zu nicht unmittelbar sichtbaren Zielen liegen soll und Rücksicht auf das Raumverständnis aller Nutzer genommen wird.
Die geeignetsten Konzepte sollen zu einem Interaktionsdesign führen, welches implementiert und durch eine Studie evaluiert wird.

