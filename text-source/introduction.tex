Mit der zunehmenden Popularisierung immersiver, virtueller Umgebungen werden die Ansprüche an verschiedene Arten, sich durch jene zu bewegen, zunehmend vielfältiger.
Für einzelne Benutzer hat sich im Konsumentenbereich die Teleportation in den meisten Fällen gegenüber sogenannter \glqq Steering Navigation\grqq{}\footnote{\textit{Steering Navigation}: Die Navigation durch den 3D-Raum mit 6 Freiheitsgraden, genauere Erklärung in Kapitel 4} durchgesetzt und ist in vielen Anwendungen, wie Computerspielen, zum Standard geworden. 
Dennoch gibt es viele Szenarien, in denen weiter gedacht werden muss, wenn es sich zum Beispiel um die Navigation über große Distanzen dreht oder die Größenverhältnisse in der Umgebung während dem Navigieren verändert werden, wobei man doch immer wieder an der Problematik der Größe des Workspaces stößt, welche an die Maße des realen Raumes, in dem man sich befindet, gebunden ist.
Dies kann zu vollständig unterschiedlichen Ansprüchen an die Navigation führen: Der Arzt, der  beispielsweise mit einer virtuellen Repräsentation eines Organs an seinem Arbeitsplatz arbeitet, braucht andere Techniken, als ein Nutzer, der sich zum Zwecke des eigenen Entertainments oder der eigenen Weiterbildung, durch die Welt von Google Earth navigiert und sich schnell zwischen den Metropolen (oder Heimatdörfern) der Welt bewegen will. Eine bis jetzt wenig betrachtete Besonderheit bei diesen verschiedenen Fortbewegungstechniken ist dabei die Navigation einer ganzen Gruppe von Nutzern, welche aber gerade in Zukunft eine wichtige Rolle in virtuellen Umgebungen spielen dürfte. Der Fremdenführer, der vor der \glqq realen\grqq{} Führung einen virtuellen Überblick über eine Stadt geben will oder ein Architekt, der mit Mitarbeitern verschiedene Standorte für ein Gebäude virtuell erkunden will, sind nur zwei Beispiele der Fülle an möglichen Anwendungsszenarien.
Ziel dieser Arbeit ist die Entwicklung und Evaluierung einer Technik, mit der es mehreren Nutzern möglich ist, gemeinsam durch virtuelle Umgebungen verschiedener Art zu reisen und dabei den wichtigsten etablierten Qualitätskriterien einer Navigationstechnik entspricht. Nach einer Einführung in die kollaborative Virtuelle Realität (VR), werden aktuelle Arbeiten zum Thema \glqq Navigation in virtuellen Umgebungen\grqq{} vorgestellt und auf ihre Anwendbarkeit in Mehrbenutzerszenarien untersucht.
Der Fokus liegt dabei auf Techniken für die effektive und bequeme Erreichbarkeit von Zielen außerhalb des sichtbaren Bereichs und ihrer Eignung zur Vermittlung von räumlichen Kenntnissen und räumlicher Orientierung. 
Die geeignetsten Konzepte sollen zu einem Interaktionsdesign führen, welches implementiert wird. Dabei wird außerdem die Abfrage der Parameter implementiert, welche für eine Studie interessant scheinen und ein Entwurf präsentiert, wie diese Studie durchzuführen wäre.

