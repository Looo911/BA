Ein zentraler Punkt, der die Nutzbarkeit der Technik noch verbessern würde, wäre die Implementierung eines Kollisionsschutzes beim Platzieren der Nutzerplattform in der WIM. Dies könnte im ersten Schritt dadurch verwirklicht werden, dass die Plattform eingefärbt wird und nicht platzierbar ist, wenn sich einer der Nutzer in bzw. hinter einer Wand oder ähnlichem befindet. Dies könnte weiterhin durch eine Art "Snapping"-Mechanismus erweitert werden, welcher die Plattform stattdessen automatisch in einer möglichen Ausrichtung dreht, die der aktuell angezeigten am nächsten ist.
Auch sind weitere Platzierungsmechanismen denkbar. So wäre z.B. eine Interaktionstechnik nach dem Vorbild der Navidget-Technik \cite{HACHET2009225} denkbar, um POI's leichter aus verschiedenen Winkeln betrachten zu können. Grundsätzlich ähneln sich die beiden Techniken nämlich in der Hinsicht des Planens einer gewünschten Kameraorientierung. Die Platzierung in der WIM bringt dabei den Vorteil mit sich, dass man auch außerhalb des aktuellen Sichtfelds navigieren kann, wobei die Ausrichtung entlang der Kugelnormalen wie bei Navidget in manchen Fällen ein Verrenken des Handgelenks verhindern kann.

Auch denkbar wäre beim Vorgang des Platzierens die Skalierung der Nutzer, die in unserem Fall immer auf eins zu eins zurückgesetzt wurde, festzulegen. Wie eine entsprechende, intuitive Festlegung der Skalierung zu bewerkstelligen wäre, müsste im Rahmen einer anderen Arbeit geklärt werden.

Bei dem Erzeugen des Portalfensters wäre außerdem denkbar, dieses, angelehnt an die Photoportale, erst am Pointer desjenigen anzuzeigen, der die Plattform in der WIM platziert hat, bevor dieser sie für alle zugänglich an die Leinwand strahlt.
Auch, wenn diese Technik im Hinblick auf Mehrbenutzerszenarien entwickelt wurde, ist eine Nutzung dieser im Einzelnutzerbereich sehr gut denkbar. So könnte ein Nutzer mit HMD auf Knopfdruck die WIM in seiner nicht-dominanten Hand aufrufen und sich selbst in dieser platzieren. Denkbar wäre dabei z.B., dass der Nutzer sich in einer transparenten "Seifenblase" befindet, welche dieser mit Hilfe des sich gerade verbreitenden Fingertrackings ganz natürlich anfassen und in die WIM stellen kann. Auch hier könnte der Nutzer mit einem Loch in dieser "Seifenblase" in die neue Welt schauen, welche auf Knopfdruck analog zu unserer Technik "aufplatzt". Im Einzelnutzerbereich mit HMD kommt einem außerdem die StepWIM\cite{Stoakley2010VirtualWIMb} wieder in den Sinn:
Hier wäre es gut denkbar, den Boden innerhalb des Trackingspaces auf Knopfdruck durch die WIM zu ersetzen, in der man sich, nachdem man zum gewünschten Reiseziel gelaufen ist, mit einer entsprechenden Selektionstechnik, ähnlich wie in unserer Implementierung, platzieren kann.
Hiermit wäre man nahe an der Implementierung von \cite{Krekhov2018GulliVRb}, würde sich allerdings den (eigentlich unnötigen) Skalierungsvorgang sparen.

Die zentrale Einschränkung der hier vorgestellten Technik stellt allerdings da, dass man aktuell nur Gebiete in Größe einer Nachbarschaft effizient bereisen kann. Eine Implementierung, mit denen man sein Ziel auch in anderen Städten und Gebieten wählen kann ist für wirklich große Distanzen unumgänglich. Wie bereits beschrieben wäre dafür die Baumtraversion von \cite{pierce_representations} oder ein Vorgehen wie bei \cite{wingrave2006overcoming} bei dem man die WIM noch besser skalier- und scrollbar macht, wäre denkbar.

